\section{Introduction}

\IEEEPARstart{T}{h}is report demonstrates how we leveraged multiple applications to explore and compare the demographic pattern and sports trend in six capital cities in Australia, including Melbourne, Sydney, Perth, Brisbane, Canberra and Adelaide. 

 The performance of application developed is improved by MRC, which provides raw IT infrastructure and allows people to develop applications with low start-up cost. By running application on MRC, the data can be extracted quicker through Twitter API with parallel running design. The way to obtain topic related data is adding sport keywords as a filter for original tweets.
 
 The analysis aims to determine the most active city by comparing demographic information from AURIN platform and Twitter data, which not only examine the current tweets trends but also use historical media data to exam the active users over time when compared sports tweets in different cities, it then yields the predictive ability in observing sports trend in the future. Taking advantage of sports challenge ranks just published recently, the most hard-working fitness users are identified by summing up challenge score of mentioned sports in user's tweets. To best utilize and compare MapReduce method, we use Spark and CouchDB cluster where Django acts as a bridge. This research also demonstrates and compares the popular sports among six cities. 
 
To best present the findings of this research, our web adds multiple interactions. For example, it enables users to locate the city by inserting a city name in a Geo-search bar, as well as the popups in each circles for each city and numbers in graphs and charts. Hopefully, this analysis will not only provide users the insight about sports trends and preference  of six capital cities but also shed light on the correlation among housing price, demographic factors, education, income, unemployment and sports trends. The code on Bitbucket enables users to test and deploy a scalable solution which can be implemented on any node of unimelb MRC to store and gather partitioned Twitter data while scaling up and down automatically.